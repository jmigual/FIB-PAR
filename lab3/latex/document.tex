\documentclass[a4paper]{article}
\usepackage[margin=2cm]{geometry}

\def\pgfsysdriver{pgfsys-pdftex.def}

\usepackage{fontspec}
\usepackage[english]{babel} % Language 
\usepackage{enumitem}
\usepackage{listings}
\usepackage[dvipsnames]{xcolor}
\usepackage{graphicx}
\usepackage{float}
\usepackage[hidelinks]{hyperref}
\usepackage{tikz}
\usepackage{pgfplots}
\usepackage[dvipsnames]{xcolor}
\usepackage{multirow}
\usepackage{caption}
\usepackage{tabls}
\usepackage{pdfpages}

\setlength{\parindent}{0pt}
\setlength{\parskip}{1em}

\lstset{
	basicstyle=\ttfamily,
	showspaces=false,
	showstringspaces=false,
	tabsize=4,
	stringstyle=\color{orange},
	commentstyle=\color{OliveGreen},
	keywordstyle=\color{blue},
	numberstyle=\color{Gray},
	frame=single
}

\title{
	\textsc{PAR: Laboratory 3} \\
	\texttt{\large par4201}
}

\author{Joan Marcè i Igual \and Esteve Tarragó i Sanchís}


\begin{document}

\maketitle
\tableofcontents
\pagebreak

\section{Task granularity analysis}

\begin{enumerate}
	\item Which are the two most important common characteristics of the task graphs generated for the two task granularities (\textit{Row} and \textit{Point}) for the non-graphical version of \texttt{mandel-tareador}? Obtain the task graphs that are generated in both cases for \texttt{-w 8}.
\end{enumerate}

For the non-graphical version of \verb|mandel-tareador| the most important charactersitc is that both tasks are parallelizable while the graphical version is not parallelizable. 

\begin{figure}[H]
	\centering
	\includegraphics[width=\textwidth]{images/depend_row_text.pdf}
	\caption{Row based task parallelization generated by \texttt{Tareador}}
\end{figure}
\begin{figure}[H]
	\centering
	\includegraphics[width=\textwidth]{images/depend_point_text.pdf}
	\caption{Point based task parallelization generated by \texttt{Tareador}}
\end{figure}

\begin{enumerate}[resume]
	\item Which section of the code is causing the serialization of all tasks in \texttt{mandeld-tareador}? How do you plan to protect this section of code in the parallel OpenMP code?
\end{enumerate}

The section of the code that is causing the serialization of all tasks in the graphic version is the following:

\begin{lstlisting}[language=C]
 if (setup_return == EXIT_SUCCESS) {
  XSetForeground (display, gc, color);
  XDrawPoint (display, win, gc, col, row);
 }           
\end{lstlisting}

To protect this section of code I will make it a critical section of the code:

\begin{lstlisting}[language=C]
if (setup_return == EXIT_SUCCESS) {
#pragma omp critical
	{
		XSetForeground (display, gc, color);
		XDrawPoint (display, win, gc, col, row);
	}
}           
\end{lstlisting}

\section{\texttt{OpenMP task-}based parallelization}

\begin{enumerate}
	\item For the \textit{Row} and \textit{Point} decompositions of the non-graphical version, include the execution time and speed–up plots obtained in the strong scalability analysis (with \texttt{-i 10000}). Reason about the causes of good or bad performance in each case.
\end{enumerate}

\section{\texttt{OpenMP for-}based parallelization}

\begin{enumerate}
	\item For the the \textit{Row} and \textit{Point} decompositions of the non-graphical version, include the execution time and speed–up plots that have been obtained for the 4 different loop schedules when using 8 threads (with \texttt{-i 10000}). Reason about the performance that is observed.
\end{enumerate}

\begin{enumerate}[resume]
	\item For the \textit{Row} parallelization strategy, complete the following table with the information extracted from the \textit{Extrae} instrumented executions (with 8 threads and \texttt{-i 10000}) and analysis with \textit{Paraver}, reasoning about the results that are obtained.
\end{enumerate}

\begin{table}[H]
	\centering
	\tablinesep=0.5cm
	\begin{tabular}{p{5cm}|rrrr}
		& \textbf{static} & \textbf{static, 10} & \textbf{dynamic, 10} & \textbf{guided, 10} \\
		\hline
		Running average time per thread & & & & \\
		Execution unbalance (average time divided per maximum time) & & & & \\
		\texttt{SchedForkJoin} (average time per thread or time if only does)
	\end{tabular}
\end{table}



\end{document}
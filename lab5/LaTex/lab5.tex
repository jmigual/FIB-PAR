\documentclass[a4paper]{article}

\usepackage[margin=2cm]{geometry}		% For desired margins
\usepackage{fontspec}					% utf-8 support
\usepackage{enumitem}					% Continuous enumerations
\usepackage{float}						% Floating images
\usepackage[dvipsnames]{xcolor}			% Colors
\usepackage{pgfplots}					% Custom plots
\usepackage{listings}					% Code highlighting
\usepackage[hidelinks]{hyperref}		% Hyper links
\usepackage{graphicx}					% To include images
\usepackage{tikz}						% To draw images
\usepackage{pgfplots}					% To draw graphs
\usepackage[dvipsnames]{xcolor	}

\pgfplotsset{compat=1.13}
\usetikzlibrary{positioning}

\setlength{\parindent}{0pt}
\setlength{\parskip}{1em}

\title{PAR: Laboratory 5 \\
		\texttt{\large par4201}}
\author{Joan Marcè i Igual \and Es6teve Tarragó i Sanchis}

\newenvironment{questionenum}{%
\setlist[enumerate]{resume}
\restartlist{enumerate}
\newcommand{\question}[1]{
\begin{enumerate}
	\item\bfseries ##1
\end{enumerate}
}}{%
}

\lstset{
	basicstyle=\small\ttfamily,
	showspaces=false,
	showstringspaces=false,
	tabsize=4,
	numbers=left,
	numbersep=5pt,
	stringstyle=\color{orange},
	commentstyle=\color{OliveGreen},
	keywordstyle=\color{blue},
	numberstyle=\color{Gray},
	frame=single
}

\begin{document}
\maketitle
\tableofcontents

\section{Analysis with Tareador}

\begin{questionenum}
	\question{Include the relevant parts of the modified \texttt{solver-tareador.c} code and comment where the calls to the Tareador API have been placed. Comment also about the task graph generated and the causes of the dependencies that appear in the two solvers: \emph{Jacobi} and \emph{Gauss-Seidel}. How will you protect them in the parallel \texttt{OpenMP} code.}
\end{questionenum}

\begin{center}	
	\begin{minipage}{0.8\textwidth}
\begin{lstlisting}[language=C, title=\texttt{heat-tareador.c}]
switch( param.algorithm ) {
	case 0: // JACOBI
		tareador_start_task("relax_jacobi");
		residual = relax_jacobi(param.u, param.uhelp, np, np);
		tareador_end_task("relax_jacobi");
		// Copy uhelp into u
		tareador_start_task("Copy_mat");
		copy_mat(param.uhelp, param.u, np, np);
		tareador_end_task("Copy_mat");
		break;
	case 1: // GAUSS
		tareador_start_task("relax_gauss");
		residual = relax_gauss(param.u, np, np);
		tareador_end_task("relax_gauss");
		break;
}
		\end{lstlisting}
	\end{minipage}
	
	\begin{minipage}{0.8\textwidth}
		\begin{lstlisting}[language=C, title=\texttt{solver-tareador.c}]
// Set tareador task
tareador_start_task("jacobi_relax");
// Disable sum object for better parallelization potential
tareador_disable_object(&sum);
utmp[i*sizey+j]= 0.25 * ( u[ i*sizey     + (j-1) ]+  // left
u[ i*sizey     + (j+1) ]+  // right
u[ (i-1)*sizey + j     ]+  // top
u[ (i+1)*sizey + j     ]); // bottom
diff = utmp[i*sizey+j] - u[i*sizey + j];
sum += diff * diff;

// Reenable sum object
tareador_enable_object(&u);

// End of tareador task
tareador_end_task("jacobi_relax");
		\end{lstlisting}
	\end{minipage}
\end{center}

\section{OpenMP parallelization and execution analysis: \emph{Jacobi}}
\begin{questionenum}
	\question{Describe the data decomposition strategy that is applied to solve the problem, including a picture with the part of the data structure that is assigned to each processor.}
	
	The decomposition strategy used is giving each thread a different part of the matrix by rows. This can be seen in the \autoref{fig:jacobi-data} where each thread has a different block of rows of the matrix to be computed. 
	
	\begin{figure}[H]
		\centering
		\includegraphics[width=0.5\textwidth]{images/jacobi/data}
		\caption{Figure describing the data decomposition, each thread has one color}
		\label{fig:jacobi-data}
	\end{figure}
	
	\question{Include the relevant portions of the parallel code that you implemented to solve the heat equation using the \emph{Jacobi} solver, commenting whatever necessary. Including captures of Paraver windows to justify your explanations and the differences observed in the execution.}
	
	\question{Include the speed-up (strong scalability) plots that have been obtained for the different numbers of processors. Reason about the performance that is observed.}
\end{questionenum}

\section{OpenMP parallelization and the execution analysis: \emph{Gauss-Seidel}}
\begin{questionenum}
	\question{Include the relevant portions of the parallel code that implements the \emph{Gauss-Seidel} solver, commenting how you implemented the synchronization between threads.}
	
	\question{Include the speed-up (strong-scalability) plot that has been obtained for the different numbers of processors. Reason about the performance that is observed, including captures of Paraver windows to justify your explanations.}
	
	\question{Explain how did you obtain the optimum value for the ration computation/synchronization in the parallelization of this solver for 8 threads.}
\end{questionenum}

\section{Optional}

\begin{questionenum}
	\question{If you have done the optional part in this laboratory assignment, please includde and comment in your report the relevant portions of the code, performance plots, or \emph{Paraver} windows that have been obtained.}
\end{questionenum}



\end{document}
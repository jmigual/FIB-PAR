\documentclass[a4paper]{article}

\usepackage[margin=2cm]{geometry}		% For desired margins
\usepackage{fontspec}					% utf-8 support
\usepackage{enumitem}					% Continuous enumerations
\usepackage{float}						% Floating images
\usepackage{pgfplots}					% Custom plots

\pgfplotsset{compat=1.13}

\title{PAR: Laboratory 4 \\
		\texttt{\large par4201}}
\author{Joan Marcè i Igual \and Esteve Tarragó i Sanchis}

\newenvironment{questionenum}
{%
\newcommand{\question}[1]{
\begin{enumerate}
	\item\bfseries ##1
\end{enumerate}
}}{%
}

\newcommand{\seamusdefault}{\textbullet}
\newcommand{\seamusx}{x}
\newcommand{\seamusy}{y}

\begin{document}
\maketitle
\tableofcontents

\section{Analysis con Tareador}
\begin{questionenum}
	\question{Pepito}
\end{questionenum}

\section{Parallelization and performance analysis with tasks}


\end{document}

Include the relevant parts of the modified \texttt{multisort-tareador.c} code and comment where the calls to the Tareador API have been placed. Comment also about the task graph generated and the causes of the dependences that appear.
\item Write a table with the execution time and speed-up predicted by \textit{Tareador} (for 1, 2, 4, 8, 16, 32 and 64 processors) for the task decomposition specified with Tareador. Are the results close to the ideal case? Reason about your answer.

\begin{enumerate}
	\item \textbf{Include the relevant portion of the codes that implement the two versions (\textit{Leaf} and \textit{Tree}), commenting whatever necessary.}
\end{enumerate}
\begin{enumerate}[resume]
	\item \textbf{For the the Leaf and Tree strategies, include the speed{up (strong scalability) plots that have been obtained for the different numbers of processors. Reason about the performance that is observed,
			including captures of Paraver windows to justify your explanations.}
	\end{enumerate}